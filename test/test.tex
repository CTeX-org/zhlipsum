\documentclass[UTF8]{ctexart}
\usepackage{kantlipsum}
\usepackage{zhlipsum}
% \usepackage{fontspec}

\begin{document}
\section{Basic}
\zhlipsum

\section{Different fonts}
{\songti   \zhlipsum[1]}

{\heiti    \zhlipsum[2]}

{\fangsong \zhlipsum[3]}

{\kaishu   \zhlipsum[4]}

\section{Single paragraph}
\zhlipsum[55]

\section{With `\textbackslash par'}
Some text before lorem ipsum.
\zhlipsum[0-1]
Some text after lorem ipsum.

\section{Without `\textbackslash par'}
Some text before lorem ipsum.
\zhlipsum*[0-500]
Some text after lorem ipsum.
\end{document}




% \documentclass{ctexart}
% \usepackage{color}
% \begin{document}
% \begin{center}
%     {\Large \bfseries 秋日登洪府滕王閣餞別序} \qquad
%     王勃
% \end{center}
% \begin{quotation}
%     \small \linespread{1.2}\selectfont
%     王勃著《滕王閣序》時年僅十四。都督閻公不之信。勃雖在座,而閻公意屬子婿孟學士者為之,已宿構矣。及以紙筆巡讓賓客,勃不辭讓。公大怒,拂衣而起,專令人伺其下筆。第一報云「豫章故郡,洪都新府」,公曰:「是亦老生談。」又報云「星分翼軫,地接衡廬」,公聞之,沈吟不言。又云「落霞與孤鶩齊飛,秋水共長天一色」,公矍然而起,曰:「此真天才,當垂不朽矣!」遂亟請宴所,極歡而罷。
% \end{quotation}
% \begin{flushright}
%     \small
%     ——《唐摭言》(卷五)
% \end{flushright}

% 豫章故郡,洪都新府。星分翼軫,地接衡廬。襟三江而帶五湖,控蠻荊而引甌越。物華天寶,龍光射牛斗之墟;人傑地靈,徐孺下陳蕃之榻。雄州霧列,俊彩星馳。臺隍枕夷夏之交,賓主盡東南之美。都督閻公之雅望,棨戟遙臨;宇文新州之懿範,襜帷暫駐。十旬休暇,勝友如雲。千里逢迎,高朋滿座。騰蛟起鳳,孟學士之詞宗;紫電青霜,王將軍之武庫。家君作宰,路出名區。童子何知?躬逢勝餞。

% 時維九月,序屬三秋。潦水盡而寒潭清,煙光凝而暮山紫。儼驂騑於上路,訪風景於崇阿。臨帝子之長洲,得仙人之舊館。層臺聳翠,上出重霄;飛閣流丹,下臨無地。鶴汀鳧渚,窮島嶼之縈廻;桂殿蘭宮,即岡巒之體勢。

% 披繡闥,俯雕甍。山原曠其盈視,川澤紆其駭矚。閭閻撲地,鐘鳴鼎食之家;舸艦迷津,青雀黃龍之舳。雲銷雨霽,彩徹區明。落霞與孤鶩齊飛,秋水共長天一色。漁舟唱晚,響窮彭蠡之濱;雁陣驚寒,聲斷衡陽之浦。

% 遙襟甫暢,逸興遄飛。爽籟發而清風生,纖歌凝而白雲遏。睢園綠竹,氣凌彭澤之樽;鄴水朱華,光照臨川之筆。四美具,二難并。窮睇眄於中天,極娛遊於暇日。天高地迥,覺宇宙之無窮;興盡悲來,識盈虛之有數。望長安於日下,目吳會於雲間。地勢極而南溟深,天柱高而北辰遠。關山難越,誰悲失路之人;萍水相逢,盡是他鄉之客。懷帝閽而不見,奉宣室以何年?

% 嗟乎!時運不齊,命途多舛。馮唐易老,李廣難封。屈賈誼於長沙,非無聖主;竄梁鴻於海曲,豈乏明時?所賴君子安貧,達人知命。老當益壯,寧移白首之心;窮且益堅,不墜青雲之志。酌貪泉而覺爽,處涸轍以猶懽。北海雖賒,扶搖可接;東隅已逝,桑榆非晚。孟嘗高潔,空餘報國之心;阮籍猖狂,豈效窮途之哭?

% 勃,三尺微命,一介書生,無路請纓,等終軍之弱冠;有懷投筆,慕宗慤之長風。捨簪笏於百齡,奉晨昏於萬里。非謝家之寶樹,接孟氏之芳鄰。他日趨庭,叨陪鯉對;今茲捧袂,喜托龍門。楊意不逢,撫凌雲而自惜;鍾期既遇,奏流水以何慚?

% 嗚呼!勝地不常,盛筵難再。蘭亭已矣,梓澤丘墟。臨別贈言,幸承恩於偉餞;登高作賦,是所望於群公。敢竭鄙誠,恭疏短引。一言均賦,四韻俱成。請灑潘江,各傾陸海云爾。

% \begin{verse}
%     滕王高閣臨江渚,佩玉鳴鸞罷歌舞。\\
%     畫棟朝飛南浦雲,珠簾暮捲西山雨。\\
%     閒雲潭影日悠悠,物換星移幾度秋。\\
%     閣中帝子今何在?檻外長江空自流!
% \end{verse}
% \end{document}